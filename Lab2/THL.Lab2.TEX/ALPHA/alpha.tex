\newpage

%%%%%%%%%%%%%%%%%  A1

\begin{justify}
    {\bf  Α.1} Να δημιουργήσετε παλμό \textlatin{SRRC} $\phi(t)$
    με τιμές  $T = 10^{-3} sec$, $over=10$, 
    $T\textsubscript{\textlatin{s}}=\frac{T}{over}$, $A=4$, και 
    $\alpha = 0.5$.
\end{justify}

\begin{justify}
    (10) Μέσω των συναρτήσεων \textlatin{fftshift} και
    \textlatin{fft}, να υπολογίσετε το μέτρο του μετασχηματισμού
    \textlatin{Fourier} της $\phi(t)$, $|\Phi(F)|$, 
    σε $N\textsubscript{\latintext{f}}$ ισαπέχοντα σημεία στο διάστημα
    $[-\frac{F\textsubscript{\textlatin{s}}}{2},\frac{F\textsubscript{\textlatin{s}}}{2}]$.
    Να σχεδιάσετε τη φασματική πυκνότητα ενέργειας $|\Phi(F)|^{2}$
    στον κατάληλλο άξονα συχνοτήτων με χρήσης της εντολής \textlatin{semiilogy}.
\end{justify}

\begin{justify}
    {\bf Λύση:}\\\\
    Έγινε χρήση της συνάρτησης $srrc\_pulse.m$ και
    επιλέγοντας τιμή $N_f=4096$ (για να μην υπάρξουν
    παραμορφώσεις).
\end{justify}

%%%%%%PLOT
\begin{center}
    \centering
    \includegraphics[width=0.8\textwidth]{ALPHA/images/a1.png} % Adjust width as neededfilename of your images
\end{center}


\begin{justify}
    και ο κώδικας \textlatin{Matlab}:
\end{justify}

\vspace{-1cm}

%%%%%%%%MATLAB code
\textlatin{
    \lstinputlisting[language=Matlab,]{ALPHA/Matlab/a1.1.m}
} 

\newpage

%%%%%%%%%%%%%%%%%  A2

\begin{justify}
    {\bf Α.2} Να δημιουργήσετε ακολουθία $N=100$ ανεξάρτητων και
    ισοπίθανων $bits \{b\textsubscript{0},...,b\textsubscript{Ν-1}\}$.
    Χρησιμοποιώντας την απεικόνιση
    \[ 0 \rightarrow +1, \]
    \[ 1 \rightarrow -1, \]
    να απεικονίσετε τα \textlatin{bits} σε σύμβολα 
    Χ\textsuperscript{\textlatin{n}} για  $n=0,...,N-1$.\\
    Να κατασκευάσετε την κυματομορφή:
    \[
        X(t) = \sum_{n=0}^{N-1} X\textsubscript{\textlatin{n}}\phi(t-nT).   
    \]
\end{justify}

\begin{justify}
    {\bf Λύση:}\\\\
    Αρχικά δημιουργήθηκε η ακολουθία \textlatin{bits} 
    και έπειτα μέσω της συνάρτησης $bits\_to\_2PAM$
    μετατράπηκαν σε \textlatin{2-PAM} σύμβολα. 
\end{justify}

%%%%%%PLOT
\begin{center}
    \centering
    \includegraphics[width=0.8\textwidth]{ALPHA/images/a2.png} % Adjust width as neededfilename of your images
\end{center}

\newpage

\begin{justify}
    και ο κώδικας \textlatin{Matlab}:
\end{justify}

\vspace{-0.8cm}

%%%%%%%%MATLAB code
\textlatin{
    \lstinputlisting[language=Matlab,]{ALPHA/Matlab/a2.m}
} 


\vspace*{2cm}


%%%%%%%%%%%%%%%%%  A3
\begin{justify}
    Υποθέτοντας ότι το πλήθος των συμβόλων είναι άπειρο, αποδείξαμε
    ότι η φασματική πυκνώτητα ισχύος του $X(t)$ είναι:
    \[
      S\textsubscript{Χ}{(F)} = \frac{\sigma^{2}\textsubscript{Χ}}{T}|\Phi(F)|^{2}.
    \]
    {\bf Α.3} (10) Με την χρήση των συναρτήσεων \textlatin{fft} και
    \textlatin{fftshift} να υπολογίσετε το περιοδόγραμμα μιάς 
    υλοποίησης της \textlatin{X(t)}:
    \[
        P\textsubscript{Χ}(F) = \frac{|F(X(t))|^{2}}{T\textsubscript{\textlatin{total}}}.  
    \]
    Να σχεδιάσετε το $P\textsubscript{Χ}(F)$ με χρήση 
    \textlatin{plot} και \textlatin{semilogy}.
\end{justify}

\newpage

\begin{justify}
    {\bf Λύση:}\\\\
    Παρακάτω παρουσιάζεται το περιοδόγραμμα $P_X(F)$
    σε δεκαδική και λογαριθμική κλίμακα:
\end{justify}

%%%%%%PLOT
\begin{center}
    \centering
    \includegraphics[width=0.8\textwidth]{ALPHA/images/a3.1.png} % Adjust width as neededfilename of your images
\end{center}


\begin{justify}
    και ο κώδικας \textlatin{Matlab}:
\end{justify}

\vspace{-0.8cm}

%%%%%%%%MATLAB code
\textlatin{
    \lstinputlisting[language=Matlab,]{ALPHA/Matlab/a2.m}
}

\newpage

\begin{justify}
    (10) Να εκτιμήσετε τη φασματική πυκνότητα ισχύος υπολογίζοντας
    αριθμιτικές μέσες τιμές πάνω σε K (ενδεικτικά, Κ=500) υλοποιήσεις
    περιοδογραμμάτων. Να σχεδιάσετε σε κοινό \textlatin{semilogy}
    την εκτίμηση και τη θεωριτική φασματική πυκνότητα ισχύος.
\end{justify}

\begin{justify}
    {\bf Λύση:}\\\\
    Έγιναν οι κατάληλλοι υπολογισμοί μέσω \textlatin{Matlab}
    και η εκτίμηση που πραγματοποιήθηκε συγκρίνεται με την
    θεωριτική φασματική πυκνότητα ισχύος η οποία προκύπτει
    από τον παραπάνω τύπο.\\\\
    Ο υπολογισμός της διασποράς $\sigma_X^2$ προκύπτει ως εξής:
    \[
        \sigma_X^2 = E[(X_n-E(X_n))^2]=E[X_n^2]=1
    \]
    Διότι: 
    \[
        E[X_n^2]=\sum x^2p_x=(1)^2\frac{1}{2}+(-1)^2\frac{1}{2}=1
    \]	
    Παρακάτω απεικονίζονται σε κοινό διάγραμμα ώστε να πραγματοποιηθεί
    η σύγκρισή τους: 
\end{justify}

%%%%%%PLOT
\begin{center}
    \centering
    \includegraphics[width=0.8\textwidth]{ALPHA/images/a3.2.png} % Adjust width as neededfilename of your images
\end{center}

\newpage

\begin{justify}
    και ο κώδικας \textlatin{Matlab}:
\end{justify}


\vspace{-0.8cm}

%%%%%%%%MATLAB code
\textlatin{
    \lstinputlisting[language=Matlab,]{ALPHA/Matlab/a3.2.m}
}

\newpage

\begin{justify}
    (10) Όσο αυξάνετε το Κ και το Ν, θα πρέπει η προσέγγιση να γίνεται
    καλύτερη.Συμβαίνει αυτό στα πειράματά σας\textlatin{;} Μπορείτε να 
    εξηγήσετε αυτό το φαινόμενο\textlatin{;}
\end{justify}

\begin{justify}
    {\bf Λύση:}\\\\
    Παρατηρούμε ότι για Κ=500 που χρησιμοποιήθηκε παραπάνω
    η προσέγγιση είναι αρεκτά καλή καθώς οι διακυμάνσεις
    των δύο γραφικών παρουσιάζουν μεγάλη ομοιότητα.\\\\
    Χρησιμοποιώντας τον ίδιο κώδικα και αλλάζοντας τις τιμές
    των παραμέτρων Ν και Κ προκύπτουν τα παρακάτω διαγράμματα:
\end{justify}

%%%%%%PLOT
\begin{center}
    \centering
    \includegraphics[width=0.8\textwidth]{ALPHA/images/a3.3.png} % Adjust width as neededfilename of your images
\end{center}

\begin{justify}
    και ο κώδικας \textlatin{Matlab}:
\end{justify}


\vspace{-0.8cm}

%%%%%%%%MATLAB code
\textlatin{
    \lstinputlisting[language=Matlab,]{ALPHA/Matlab/a3.3.m}
}


\begin{justify}
    Όπως επιβεβαιώνεται και από τα παραπάνω διαγράμματα, όσο
    αυξάνετε το Κ και το Ν, η προσέγγιση γίνεται καλύτερη.
    Αυτο συμβαίνει διότι τα περιοδογράμματα ακολουθούν κανονική
    κατανομή, όπως και τα \textlatin{bits} τα οποία
    κωδικοποιούνται και καταλήγουμε στην $X(t)$, οπότε
    όσο μεγαλύτερο γίνεται το δείγμα τόσο καλύτερη και πιο ακριβής
    γίνεται	η προσέγγιση.
\end{justify}

\newpage

%%%%%%%%%%%%%%%%%  A4

\begin{justify}
    {\bf A.4} Χρησιμοποιώντας την απεικόνιση:
    \[
      00 \rightarrow +3, 
    \]
    \[
      01 \rightarrow +1,
    \]
    \[
      11 \rightarrow -1,
    \]
    \[
      10 \rightarrow -3, 
    \]
    να κατασκευάσετε την ακολουθία 4-\textlatin{PAM} $X\textsubscript{\latintext{n}}$,
    για $n=0,...,\frac{N}{2}-1$.
\end{justify}

\begin{justify}
    {\bf Λύση:}\\\\
    Αρχικά κατασκευάστηκε η συνάρτηση $bits\_to\_4PAM$: 
\end{justify}

\vspace{-0.8cm}

%%%%%%%%MATLAB code
\textlatin{
    \lstinputlisting[language=Matlab,]{ALPHA/Matlab/bits_to4PAM.m}
}


\newpage

\begin{justify}
    Να κατασκευάσετε την κυματομορφή:
    \[
        X(t) = \sum_{n=0}^{\frac{N}{2}-1} X\textsubscript{\textlatin{n}}\phi(t-nT). 
    \]
    χρησιμοποιώντας την ίδια περίοδο $T$ με το ερώτημα {\bf Α.2}.
\end{justify}


\begin{justify}
    {\bf Λύση:}\\\\
    Όμοια με το ερώτημα {\bf Α.2}:
\end{justify}

%%%%%%PLOT
\begin{center}
    \centering
    \includegraphics[width=0.8\textwidth]{ALPHA/images/a4.png} % Adjust width as neededfilename of your images
\end{center}


\begin{justify}
    (10) Να υπολογίσετε το περιοδόγραμμα και να εκτιμήσετε τη φασματική πυκνότητα
    ισχύος μέσω αριθμητικών μέσων τιμών υλοποιήσεων περιοδογραμμάτων της $X(\textlatin{t})$. Να
    σχεδιάσετε την πειραματική και την θεωρητική φασματική πυκνότητα ισχύος στο ίδιο
    \textlatin{semilogy}. Τι παρατηρείτε\textlatin{;}
\end{justify}

\begin{justify}
    {\bf Λύση:}\\\\
    Παρομοίως με το ερώτημα {\bf Α.3}, πραγματοποιήθηκε η εκτίμηση
    της φασματικής πυκνότητας ισχύος. Έπειτα, για να βρεθεί η 
    αντίστοιχη θεωρητική φασματική πυκνότητα ισχύος, ξαναχρησιμοποιήθηκε
    ο τύπος του {\bf Α.3}, με την διαφορά ότι η νέα
    διασπορά $\sigma_X^2$ είναι:
    \[
         \sigma_X^2 = E(X_n^2)=\sum x^2p_x=(1)^2\frac{1}{4}+(-1)^2\frac{1}{4}
         +(3)^2\frac{1}{4}+(-3)^2\frac{1}{4}=5
    \]
    Τέλος σχεδιάστηκαν και οι δύο σε ένα κοινό διάγραμμα με
    λογαριθμική κλίμακα.
\end{justify}

%%%%%%PLOT
\begin{center}
    \centering
    \includegraphics[width=0.8\textwidth]{ALPHA/images/a4.1.png} % Adjust width as neededfilename of your images
\end{center}



\begin{justify}
    Παρατηρούμε όμοια συμπεριφορά με το ερώτημα {\bf Α.3}, για το πλήθος
    των περιοδογραμμάτων Κ=500 που χρησιμοποιήθηκαν για την εκτίμηση.
\end{justify}

\begin{justify}
    και ο κώδικας \textlatin{Matlab}:
\end{justify}

\vspace{-0.8cm}

%%%%%%%%MATLAB code
\textlatin{
    \lstinputlisting[language=Matlab,]{ALPHA/Matlab/a4.1.m}
}



\begin{justify}
    (10) Πώς συγκρίνεται, ως προς το εύρος φάσματος και ως προς το μέγιστο πλάτος τιμών,
    η φασματική πυκνότητα ισχύος της $X(\textlatin{t})$ σε σχέση με αυτή της $X(\textlatin{t})$ του βήματος {\bf A.2}\textlatin{;}
    Μπορείτε να εξηγήσετε τα αποτελέσματα της σύγκρισης\textlatin{;}
\end{justify}

%%%%%%PLOT
\begin{center}
    \centering
    \includegraphics[width=0.8\textwidth]{ALPHA/images/a4.2.png} % Adjust width as neededfilename of your images
\end{center}


\begin{justify}
    {\bf Λύση:}\\\\
    Απεικονίσαμε σε κοινό \textlatin{plot} και \textlatin{semilogy}
    τις φασματικές πυκνότητες ισχύος 2-\textlatin{PAM} και
    4-\textlatin{PAM}. Παρατηρούμε ότι το εύρος φάσματος και των δύο
    είναι ίδιο αφού $BW=\frac{1+\alpha}{2Τ}$, το οποίο εξαρτάται μόνο
    από την συχνότητα $T$ και το $\alpha$. Στην συνέχεια, Παρατηρούμε
    ότι το μέγιστο πλάτος των τιμών της 4-\textlatin{PAM} είναι σχεδόν
    5 φορές μεγαλύτερο έναντι με αυτού της 2-\textlatin{PAM} και αυτό
    συμβαίνει διότι η διασπορά της 4-\textlatin{PAM} είναι μεγαλύτερη.
\end{justify}

\begin{justify}
    και ο κώδικας \textlatin{Matlab}:
\end{justify}

\vspace{-0.8cm}

%%%%%%%%MATLAB code
\textlatin{
    \lstinputlisting[language=Matlab,]{ALPHA/Matlab/a4.2.m}
}

\newpage

%%%%%%%%%%%%%%%%%  A5

\begin{justify}
    {\bf A.5} (10) Να επαναλάβετε το βήμα Α.3, θέτοντας περίοδο συμβόλου $T' = 2T$ (να διατηρήσετε
    την περίοδο δειγματοληψίας $T_s$ ίση με αυτή των προηγούμενων βημάτων, άρα, θα πρέπει
    να διπλασιάσετε την παράμετρο \textlatin{over}).
\end{justify}

\begin{justify}
    {\bf Λύση:}\\\\
    Ακολουθήθηκε η ίδια διαδικασία με το ερώτημα {\bf Α.3}, 
    με $Τ'=2\cdot10^-3$ και $over'=20$. Παρακάτω παρουσιάζεται το 
    περιοδόγραμμα $P_X(F)$ σε δεκαδική και λογαριθμική κλίμακα:
\end{justify}

%%%%%%PLOT
\begin{center}
    \centering
    \includegraphics[width=0.8\textwidth]{ALPHA/images/a5.1.png} % Adjust width as neededfilename of your images
\end{center}

\vspace{-0.8cm}

%%%%%%%%MATLAB code
\textlatin{
    \lstinputlisting[language=Matlab,]{ALPHA/Matlab/a5.1.m}
}


\begin{justify}
    Και  όμοια πάλι με το ερώτημα
    {\bf Α.3}, προκύπτει το κοινό διάγραμμα για την θεωρητική και την 
    εκτιμόμενη φασματική πυκνότητα ισχύος:
\end{justify}


%%%%%%PLOT
\begin{center}
    \centering
    \includegraphics[width=0.6\textwidth]{ALPHA/images/a5.2.png} % Adjust width as neededfilename of your images
\end{center}

\vspace{-0.8cm}

%%%%%%%%MATLAB code
\textlatin{
    \lstinputlisting[language=Matlab,]{ALPHA/Matlab/a5.2.m}
}

\newpage

\begin{justify}
    Τέλος χρησιμοποιώντας διαφορετικά Κ και Ν προκύπτουν τα παρακάτω
    διαγράμματα:
\end{justify}

%%%%%%PLOT
\begin{center}
    \centering
    \includegraphics[width=0.8\textwidth]{ALPHA/images/a5.3.png} % Adjust width as neededfilename of your images
\end{center}

\begin{justify}
    (5) Τι παρατηρείτε σχετικά με το εύρος φάσματος 
    των κυματομορφών σε αυτή την
    περίπτωση σε σχέση με αυτό των κυματομορφών 
    του βήματος Α.3\textlatin{;} Μπορείτε να εξηγήσετε το φαινόμενο\textlatin{;}
\end{justify}

\begin{justify}
    {\bf Λύση:}\\\\
    Κάνοντας κοινό \textlatin{plot} των περιοδογραμμάτων των 2
    ερωτημάτων:
\end{justify}

%%%%%%PLOT
\begin{center}
    \centering
    \includegraphics[width=0.6\textwidth]{ALPHA/images/a5.4.png} % Adjust width as neededfilename of your images
\end{center}



\begin{justify}
    Γίνεται εύκολα αντιληπτό ότι για $T'=2T$ μείωνεται εμφανώς το
    το εύρος φάσματος, πράγμα που επιβεβαιώνεται και από την θεωρία αφού
    $BW=\frac{1+\alpha}{2Τ}$.
\end{justify}

\newpage

%%%%%%%%%%%%%%%%%%%%%%%%%%%%%%  A6

\begin{justify}
    {\bf Α.6} (2.5) Αν θέλατε να στείλετε δεδομένα
    όσο το δυνατό ταχύτερα έχοντας διαθέσιμο το ίδιο εύρος
    φάσματος, θα επιλέγατε \textlatin{2-PAM} ή \textlatin{4-PAM},
    και γιατί\textlatin{;}    
\end{justify}

\begin{justify}
    {\bf Λύση:}\\\\
    Για να επιτευχθεί η ταχύτερη διάδοση δεδομένων έχοντας διαθέσιμο
    το ίδιο εύρος φάσματος, θα επιλέγαμε \textlatin{4-PAM} καθώς 
    αντιστοιχίζονται δύο \textlatin{bits} ανα σύμβολο σε αντίθεση
    με το \textlatin{2-PAM} που αντιστοιχίζεται ένα. Επομένως στέλνονται
    περισσότερα \textlatin{bits} πληροφορίας στον ίδιο χρόνο.
\end{justify}

\begin{justify}
    (2.5) Αν το διαθέσιμο εύρος φάσματος είναι πολύ ακριβό,
    θα επιλέγατε περίοδο συμβόλου $Τ$ ή $Τ'=2T$, και γιατί\textlatin{;}
\end{justify}

\begin{justify}
    {\bf Λύση:}\\\\
    Αν το διαθέσιμο εύρος φάσματος είναι πολύ ακριβό, θα επιλέγαμε
    περίοδο συμβόλου $Τ'=2T$ καθώς, όπως δείξαμε και στο ερώτημα
    {\bf Α.5}, θα έδινε μικρότερο εύρος φάσματος και έτσι θα εξοικονομούνταν
    στην μετάδοση το μισό φάσμα.
\end{justify}

